%% Generated by Sphinx.
\def\sphinxdocclass{report}
\documentclass[letterpaper,10pt,english]{sphinxmanual}
\ifdefined\pdfpxdimen
   \let\sphinxpxdimen\pdfpxdimen\else\newdimen\sphinxpxdimen
\fi \sphinxpxdimen=.75bp\relax

\PassOptionsToPackage{warn}{textcomp}
\usepackage[utf8]{inputenc}
\ifdefined\DeclareUnicodeCharacter
% support both utf8 and utf8x syntaxes
  \ifdefined\DeclareUnicodeCharacterAsOptional
    \def\sphinxDUC#1{\DeclareUnicodeCharacter{"#1}}
  \else
    \let\sphinxDUC\DeclareUnicodeCharacter
  \fi
  \sphinxDUC{00A0}{\nobreakspace}
  \sphinxDUC{2500}{\sphinxunichar{2500}}
  \sphinxDUC{2502}{\sphinxunichar{2502}}
  \sphinxDUC{2514}{\sphinxunichar{2514}}
  \sphinxDUC{251C}{\sphinxunichar{251C}}
  \sphinxDUC{2572}{\textbackslash}
\fi
\usepackage{cmap}
\usepackage[T1]{fontenc}
\usepackage{amsmath,amssymb,amstext}
\usepackage{babel}



\usepackage{times}
\expandafter\ifx\csname T@LGR\endcsname\relax
\else
% LGR was declared as font encoding
  \substitutefont{LGR}{\rmdefault}{cmr}
  \substitutefont{LGR}{\sfdefault}{cmss}
  \substitutefont{LGR}{\ttdefault}{cmtt}
\fi
\expandafter\ifx\csname T@X2\endcsname\relax
  \expandafter\ifx\csname T@T2A\endcsname\relax
  \else
  % T2A was declared as font encoding
    \substitutefont{T2A}{\rmdefault}{cmr}
    \substitutefont{T2A}{\sfdefault}{cmss}
    \substitutefont{T2A}{\ttdefault}{cmtt}
  \fi
\else
% X2 was declared as font encoding
  \substitutefont{X2}{\rmdefault}{cmr}
  \substitutefont{X2}{\sfdefault}{cmss}
  \substitutefont{X2}{\ttdefault}{cmtt}
\fi


\usepackage[Bjarne]{fncychap}
\usepackage{sphinx}

\fvset{fontsize=\small}
\usepackage{geometry}


% Include hyperref last.
\usepackage{hyperref}
% Fix anchor placement for figures with captions.
\usepackage{hypcap}% it must be loaded after hyperref.
% Set up styles of URL: it should be placed after hyperref.
\urlstyle{same}
\addto\captionsenglish{\renewcommand{\contentsname}{Contents:}}

\usepackage{sphinxmessages}
\setcounter{tocdepth}{1}


    \hypersetup{unicode=true}
    \usepackage{CJKutf8}
    \AtBeginDocument{\begin{CJK}{UTF8}{gbsn}}
    \AtEndDocument{\end{CJK}}
    

\title{learn\sphinxhyphen{}sphinx}
\date{Jan 10, 2020}
\release{1.0}
\author{pku}
\newcommand{\sphinxlogo}{\vbox{}}
\renewcommand{\releasename}{Release}
\makeindex
\begin{document}

\pagestyle{empty}
\sphinxmaketitle
\pagestyle{plain}
\sphinxtableofcontents
\pagestyle{normal}
\phantomsection\label{\detokenize{index::doc}}



\chapter{Getting started}
\label{\detokenize{chapter1:getting-started}}\label{\detokenize{chapter1:id1}}\label{\detokenize{chapter1::doc}}

\section{Installing your doc directory}
\label{\detokenize{chapter1:installing-your-doc-directory}}\label{\detokenize{chapter1:installing-docdir}}
You may already have sphinx \sphinxhref{http://sphinx.pocoo.org/}{sphinx}
installed \textendash{} you can check by doing:

\begin{sphinxVerbatim}[commandchars=\\\{\}]
\PYG{n}{python} \PYG{o}{\PYGZhy{}}\PYG{n}{c} \PYG{l+s+s1}{\PYGZsq{}}\PYG{l+s+s1}{import sphinx}\PYG{l+s+s1}{\PYGZsq{}}
\end{sphinxVerbatim}

If that fails grab the latest version of and install it with:

\begin{sphinxVerbatim}[commandchars=\\\{\}]
\PYG{o}{\PYGZgt{}} \PYG{n}{sudo} \PYG{n}{easy\PYGZus{}install} \PYG{o}{\PYGZhy{}}\PYG{n}{U} \PYG{n}{Sphinx}
\end{sphinxVerbatim}

Now you are ready to build a template for your docs, using
sphinx\sphinxhyphen{}quickstart:

\begin{sphinxVerbatim}[commandchars=\\\{\}]
\PYG{o}{\PYGZgt{}} \PYG{n}{sphinx}\PYG{o}{\PYGZhy{}}\PYG{n}{quickstart}
\end{sphinxVerbatim}

accepting most of the defaults.  I choose “sampledoc” as the name of my
project.  cd into your new directory and check the contents:

\begin{sphinxVerbatim}[commandchars=\\\{\}]
\PYG{n}{home}\PYG{p}{:}\PYG{o}{\PYGZti{}}\PYG{o}{/}\PYG{n}{tmp}\PYG{o}{/}\PYG{n}{sampledoc}\PYG{o}{\PYGZgt{}} \PYG{n}{ls}
\PYG{n}{Makefile}      \PYG{n}{\PYGZus{}static}         \PYG{n}{conf}\PYG{o}{.}\PYG{n}{py}
\PYG{n}{build}         \PYG{n}{\PYGZus{}templates}      \PYG{n}{index}\PYG{o}{.}\PYG{n}{rst}
\end{sphinxVerbatim}

The index.rst is the master ReST for your project, but before adding
anything, let’s see if we can build some html:

\begin{sphinxVerbatim}[commandchars=\\\{\}]
\PYG{n}{make} \PYG{n}{html}
\end{sphinxVerbatim}

If you now point your browser to \sphinxcode{\sphinxupquote{build/html/index.html}}, you
should see a basic sphinx site.

\noindent\sphinxincludegraphics{{images/basic_screenshot}.png}


\subsection{Fetching the data}
\label{\detokenize{chapter1:fetching-the-data}}\label{\detokenize{chapter1:id2}}
Now we will start to customize out docs.  Grab a couple of files from
the \sphinxhref{https://github.com/matplotlib/sampledoc}{web site}
or git.  You will need \sphinxcode{\sphinxupquote{getting\_started.rst}} and
\sphinxcode{\sphinxupquote{images/basic\_screenshot.png}}.  All of the files live in the
“completed” version of this tutorial, but since this is a tutorial,
we’ll just grab them one at a time, so you can learn what needs to be
changed where.  Since we have more files to come, I’m going to grab
the whole git directory and just copy the files I need over for now.
First, I’ll cd up back into the directory containing my project, check
out the “finished” product from git, and then copy in just the files I
need into my \sphinxcode{\sphinxupquote{sampledoc}} directory:

\begin{sphinxVerbatim}[commandchars=\\\{\}]
\PYG{n}{home}\PYG{p}{:}\PYG{o}{\PYGZti{}}\PYG{o}{/}\PYG{n}{tmp}\PYG{o}{/}\PYG{n}{sampledoc}\PYG{o}{\PYGZgt{}} \PYG{n}{pwd}
\PYG{o}{/}\PYG{n}{Users}\PYG{o}{/}\PYG{n}{jdhunter}\PYG{o}{/}\PYG{n}{tmp}\PYG{o}{/}\PYG{n}{sampledoc}
\PYG{n}{home}\PYG{p}{:}\PYG{o}{\PYGZti{}}\PYG{o}{/}\PYG{n}{tmp}\PYG{o}{/}\PYG{n}{sampledoc}\PYG{o}{\PYGZgt{}} \PYG{n}{cd} \PYG{o}{.}\PYG{o}{.}
\PYG{n}{home}\PYG{p}{:}\PYG{o}{\PYGZti{}}\PYG{o}{/}\PYG{n}{tmp}\PYG{o}{\PYGZgt{}} \PYG{n}{git} \PYG{n}{clone} \PYG{n}{https}\PYG{p}{:}\PYG{o}{/}\PYG{o}{/}\PYG{n}{github}\PYG{o}{.}\PYG{n}{com}\PYG{o}{/}\PYG{n}{matplotlib}\PYG{o}{/}\PYG{n}{sampledoc}\PYG{o}{.}\PYG{n}{git} \PYG{n}{tutorial}
\PYG{n}{Cloning} \PYG{n}{into} \PYG{l+s+s1}{\PYGZsq{}}\PYG{l+s+s1}{tutorial}\PYG{l+s+s1}{\PYGZsq{}}\PYG{o}{.}\PYG{o}{.}\PYG{o}{.}
\PYG{n}{remote}\PYG{p}{:} \PYG{n}{Counting} \PYG{n}{objects}\PYG{p}{:} \PYG{l+m+mi}{87}\PYG{p}{,} \PYG{n}{done}\PYG{o}{.}
\PYG{n}{remote}\PYG{p}{:} \PYG{n}{Compressing} \PYG{n}{objects}\PYG{p}{:} \PYG{l+m+mi}{100}\PYG{o}{\PYGZpc{}} \PYG{p}{(}\PYG{l+m+mi}{43}\PYG{o}{/}\PYG{l+m+mi}{43}\PYG{p}{)}\PYG{p}{,} \PYG{n}{done}\PYG{o}{.}
\PYG{n}{remote}\PYG{p}{:} \PYG{n}{Total} \PYG{l+m+mi}{87} \PYG{p}{(}\PYG{n}{delta} \PYG{l+m+mi}{45}\PYG{p}{)}\PYG{p}{,} \PYG{n}{reused} \PYG{l+m+mi}{83} \PYG{p}{(}\PYG{n}{delta} \PYG{l+m+mi}{41}\PYG{p}{)}
\PYG{n}{Unpacking} \PYG{n}{objects}\PYG{p}{:} \PYG{l+m+mi}{100}\PYG{o}{\PYGZpc{}} \PYG{p}{(}\PYG{l+m+mi}{87}\PYG{o}{/}\PYG{l+m+mi}{87}\PYG{p}{)}\PYG{p}{,} \PYG{n}{done}\PYG{o}{.}
\PYG{n}{Checking} \PYG{n}{connectivity}\PYG{o}{.}\PYG{o}{.}\PYG{o}{.} \PYG{n}{done}
\PYG{n}{home}\PYG{p}{:}\PYG{o}{\PYGZti{}}\PYG{o}{/}\PYG{n}{tmp}\PYG{o}{\PYGZgt{}} \PYG{n}{cp} \PYG{n}{tutorial}\PYG{o}{/}\PYG{n}{getting\PYGZus{}started}\PYG{o}{.}\PYG{n}{rst} \PYG{n}{sampledoc}\PYG{o}{/}
\PYG{n}{home}\PYG{p}{:}\PYG{o}{\PYGZti{}}\PYG{o}{/}\PYG{n}{tmp}\PYG{o}{\PYGZgt{}} \PYG{n}{cp} \PYG{n}{tutorial}\PYG{o}{/}\PYG{n}{\PYGZus{}static}\PYG{o}{/}\PYG{n}{basic\PYGZus{}screenshot}\PYG{o}{.}\PYG{n}{png} \PYG{n}{sampledoc}\PYG{o}{/}\PYG{n}{\PYGZus{}static}\PYG{o}{/}
\end{sphinxVerbatim}

The last step is to modify \sphinxcode{\sphinxupquote{index.rst}} to include the
\sphinxcode{\sphinxupquote{getting\_started.rst}} file (be careful with the indentation, the
“g” in “getting\_started” should line up with the ‘:’ in \sphinxcode{\sphinxupquote{:maxdepth}}:

\begin{sphinxVerbatim}[commandchars=\\\{\}]
\PYG{n}{Contents}\PYG{p}{:}

\PYG{o}{.}\PYG{o}{.} \PYG{n}{toctree}\PYG{p}{:}\PYG{p}{:}
   \PYG{p}{:}\PYG{n}{maxdepth}\PYG{p}{:} \PYG{l+m+mi}{2}

   \PYG{n}{getting\PYGZus{}started}\PYG{o}{.}\PYG{n}{rst}
\end{sphinxVerbatim}

and then rebuild the docs:

\begin{sphinxVerbatim}[commandchars=\\\{\}]
\PYG{n}{cd} \PYG{n}{sampledoc}
\PYG{n}{make} \PYG{n}{html}
\end{sphinxVerbatim}

When you reload the page by refreshing your browser pointing to
\sphinxcode{\sphinxupquote{build/html/index.html}}, you should see a link to the
“Getting Started” docs, and in there this page with the screenshot.
\sphinxtitleref{Voila!}

We can also use the image directive in \sphinxcode{\sphinxupquote{index.rst}} to include to the screenshot above
with:

\begin{sphinxVerbatim}[commandchars=\\\{\}]
\PYG{o}{.}\PYG{o}{.} \PYG{n}{image}\PYG{p}{:}\PYG{p}{:}
   \PYG{n}{images}\PYG{o}{/}\PYG{n}{basic\PYGZus{}screenshot}\PYG{o}{.}\PYG{n}{png}
\end{sphinxVerbatim}

Next we’ll customize the look and feel of our site to give it a logo,
some custom css, and update the navigation panels to look more like
the \sphinxhref{http://sphinx.pocoo.org/}{sphinx} site itself \textendash{} see
\DUrole{xref,std,std-ref}{custom\_look}.


\chapter{Sphinx extensions for embedded plots, math and more}
\label{\detokenize{chapter2:sphinx-extensions-for-embedded-plots-math-and-more}}\label{\detokenize{chapter2:extensions}}\label{\detokenize{chapter2::doc}}
Sphinx is written in python, and supports the ability to write custom
extensions.  We’ve written a few for the matplotlib documentation,
some of which are part of matplotlib itself in the
matplotlib.sphinxext module, some of which are included only in the
sphinx doc directory, and there are other extensions written by other
groups, eg numpy and ipython.  We’re collecting these in this tutorial
and showing you how to install and use them for your own project.
First let’s grab the python extension files from the \sphinxcode{\sphinxupquote{sphinxext}}
directory from git (see {\hyperref[\detokenize{chapter1:fetching-the-data}]{\sphinxcrossref{\DUrole{std,std-ref}{Fetching the data}}}}), and install them in
our \sphinxcode{\sphinxupquote{sampledoc}} project \sphinxcode{\sphinxupquote{sphinxext}} directory:

\begin{sphinxVerbatim}[commandchars=\\\{\}]
\PYG{n}{home}\PYG{p}{:}\PYG{o}{\PYGZti{}}\PYG{o}{/}\PYG{n}{tmp}\PYG{o}{/}\PYG{n}{sampledoc}\PYG{o}{\PYGZgt{}} \PYG{n}{mkdir} \PYG{n}{sphinxext}
\PYG{n}{home}\PYG{p}{:}\PYG{o}{\PYGZti{}}\PYG{o}{/}\PYG{n}{tmp}\PYG{o}{/}\PYG{n}{sampledoc}\PYG{o}{\PYGZgt{}} \PYG{n}{cp} \PYG{o}{.}\PYG{o}{.}\PYG{o}{/}\PYG{n}{sampledoc\PYGZus{}tut}\PYG{o}{/}\PYG{n}{sphinxext}\PYG{o}{/}\PYG{o}{*}\PYG{o}{.}\PYG{n}{py} \PYG{n}{sphinxext}\PYG{o}{/}
\PYG{n}{home}\PYG{p}{:}\PYG{o}{\PYGZti{}}\PYG{o}{/}\PYG{n}{tmp}\PYG{o}{/}\PYG{n}{sampledoc}\PYG{o}{\PYGZgt{}} \PYG{n}{ls} \PYG{n}{sphinxext}\PYG{o}{/}
\PYG{n}{apigen}\PYG{o}{.}\PYG{n}{py}  \PYG{n}{docscrape}\PYG{o}{.}\PYG{n}{py}  \PYG{n}{docscrape\PYGZus{}sphinx}\PYG{o}{.}\PYG{n}{py}  \PYG{n}{numpydoc}\PYG{o}{.}\PYG{n}{py}
\end{sphinxVerbatim}

In addition to the builtin matplotlib extensions for embedding pyplot
plots and rendering math with matplotlib’s native math engine, we also
have extensions for syntax highlighting ipython sessions, making
inhertiance diagrams, and more.

We need to inform sphinx of our new extensions in the \sphinxcode{\sphinxupquote{conf.py}}
file by adding the following.  First we tell it where to find the extensions:

\begin{sphinxVerbatim}[commandchars=\\\{\}]
\PYG{c+c1}{\PYGZsh{} If your extensions are in another directory, add it here. If the}
\PYG{c+c1}{\PYGZsh{} directory is relative to the documentation root, use}
\PYG{c+c1}{\PYGZsh{} os.path.abspath to make it absolute, like shown here.}
\PYG{n}{sys}\PYG{o}{.}\PYG{n}{path}\PYG{o}{.}\PYG{n}{append}\PYG{p}{(}\PYG{n}{os}\PYG{o}{.}\PYG{n}{path}\PYG{o}{.}\PYG{n}{abspath}\PYG{p}{(}\PYG{l+s+s1}{\PYGZsq{}}\PYG{l+s+s1}{sphinxext}\PYG{l+s+s1}{\PYGZsq{}}\PYG{p}{)}\PYG{p}{)}
\end{sphinxVerbatim}

And then we tell it what extensions to load:

\begin{sphinxVerbatim}[commandchars=\\\{\}]
\PYG{c+c1}{\PYGZsh{} Add any Sphinx extension module names here, as strings. They can be extensions}
\PYG{c+c1}{\PYGZsh{} coming with Sphinx (named \PYGZsq{}sphinx.ext.*\PYGZsq{}) or your custom ones.}
\PYG{n}{extensions} \PYG{o}{=} \PYG{p}{[}\PYG{l+s+s1}{\PYGZsq{}}\PYG{l+s+s1}{matplotlib.sphinxext.only\PYGZus{}directives}\PYG{l+s+s1}{\PYGZsq{}}\PYG{p}{,}
              \PYG{l+s+s1}{\PYGZsq{}}\PYG{l+s+s1}{matplotlib.sphinxext.plot\PYGZus{}directive}\PYG{l+s+s1}{\PYGZsq{}}\PYG{p}{,}
              \PYG{l+s+s1}{\PYGZsq{}}\PYG{l+s+s1}{IPython.sphinxext.ipython\PYGZus{}directive}\PYG{l+s+s1}{\PYGZsq{}}\PYG{p}{,}
              \PYG{l+s+s1}{\PYGZsq{}}\PYG{l+s+s1}{IPython.sphinxext.ipython\PYGZus{}console\PYGZus{}highlighting}\PYG{l+s+s1}{\PYGZsq{}}\PYG{p}{,}
              \PYG{l+s+s1}{\PYGZsq{}}\PYG{l+s+s1}{sphinx.ext.mathjax}\PYG{l+s+s1}{\PYGZsq{}}\PYG{p}{,}
              \PYG{l+s+s1}{\PYGZsq{}}\PYG{l+s+s1}{sphinx.ext.autodoc}\PYG{l+s+s1}{\PYGZsq{}}\PYG{p}{,}
              \PYG{l+s+s1}{\PYGZsq{}}\PYG{l+s+s1}{sphinx.ext.doctest}\PYG{l+s+s1}{\PYGZsq{}}\PYG{p}{,}
              \PYG{l+s+s1}{\PYGZsq{}}\PYG{l+s+s1}{sphinx.ext.inheritance\PYGZus{}diagram}\PYG{l+s+s1}{\PYGZsq{}}\PYG{p}{,}
              \PYG{l+s+s1}{\PYGZsq{}}\PYG{l+s+s1}{numpydoc}\PYG{l+s+s1}{\PYGZsq{}}\PYG{p}{]}
\end{sphinxVerbatim}

Now let’s look at some of these in action.  You can see the literal
source for this file at {\hyperref[\detokenize{chapter2:extensions-literal}]{\sphinxcrossref{\DUrole{std,std-ref}{This file}}}}.


\section{ipython sessions}
\label{\detokenize{chapter2:ipython-sessions}}\label{\detokenize{chapter2:ipython-highlighting}}
Michael Droettboom contributed a sphinx extension which does \sphinxhref{http://pygments.org}{pygments} syntax highlighting on \sphinxhref{http://ipython.scipy.org}{ipython} sessions.  Just use ipython as the
language in the \sphinxcode{\sphinxupquote{sourcecode}} directive:

\begin{sphinxVerbatim}[commandchars=\\\{\}]
\PYG{o}{.}\PYG{o}{.} \PYG{n}{sourcecode}\PYG{p}{:}\PYG{p}{:} \PYG{n}{ipython}

    \PYG{n}{In} \PYG{p}{[}\PYG{l+m+mi}{69}\PYG{p}{]}\PYG{p}{:} \PYG{n}{lines} \PYG{o}{=} \PYG{n}{plot}\PYG{p}{(}\PYG{p}{[}\PYG{l+m+mi}{1}\PYG{p}{,}\PYG{l+m+mi}{2}\PYG{p}{,}\PYG{l+m+mi}{3}\PYG{p}{]}\PYG{p}{)}

    \PYG{n}{In} \PYG{p}{[}\PYG{l+m+mi}{70}\PYG{p}{]}\PYG{p}{:} \PYG{n}{setp}\PYG{p}{(}\PYG{n}{lines}\PYG{p}{)}
      \PYG{n}{alpha}\PYG{p}{:} \PYG{n+nb}{float}
      \PYG{n}{animated}\PYG{p}{:} \PYG{p}{[}\PYG{k+kc}{True} \PYG{o}{|} \PYG{k+kc}{False}\PYG{p}{]}
      \PYG{n}{antialiased} \PYG{o+ow}{or} \PYG{n}{aa}\PYG{p}{:} \PYG{p}{[}\PYG{k+kc}{True} \PYG{o}{|} \PYG{k+kc}{False}\PYG{p}{]}
      \PYG{o}{.}\PYG{o}{.}\PYG{o}{.}\PYG{n}{snip}
\end{sphinxVerbatim}

and you will get the syntax highlighted output below.

\begin{sphinxVerbatim}[commandchars=\\\{\}]
\PYG{n}{In} \PYG{p}{[}\PYG{l+m+mi}{69}\PYG{p}{]}\PYG{p}{:} \PYG{n}{lines} \PYG{o}{=} \PYG{n}{plot}\PYG{p}{(}\PYG{p}{[}\PYG{l+m+mi}{1}\PYG{p}{,}\PYG{l+m+mi}{2}\PYG{p}{,}\PYG{l+m+mi}{3}\PYG{p}{]}\PYG{p}{)}

\PYG{n}{In} \PYG{p}{[}\PYG{l+m+mi}{70}\PYG{p}{]}\PYG{p}{:} \PYG{n}{setp}\PYG{p}{(}\PYG{n}{lines}\PYG{p}{)}
  \PYG{n}{alpha}\PYG{p}{:} \PYG{n+nb}{float}
  \PYG{n}{animated}\PYG{p}{:} \PYG{p}{[}\PYG{n+nb+bp}{True} \PYG{o}{|} \PYG{n+nb+bp}{False}\PYG{p}{]}
  \PYG{n}{antialiased} \PYG{o+ow}{or} \PYG{n}{aa}\PYG{p}{:} \PYG{p}{[}\PYG{n+nb+bp}{True} \PYG{o}{|} \PYG{n+nb+bp}{False}\PYG{p}{]}
  \PYG{o}{.}\PYG{o}{.}\PYG{o}{.}\PYG{n}{snip}
\end{sphinxVerbatim}

This support is included in this template, but will also be included
in a future version of Pygments by default.


\section{Using math}
\label{\detokenize{chapter2:using-math}}\label{\detokenize{chapter2:id1}}
In sphinx you can include inline math \(x\leftarrow y\ x\forall
y\ x-y\) or display math
\begin{equation*}
\begin{split}W^{3\beta}_{\delta_1 \rho_1 \sigma_2} = U^{3\beta}_{\delta_1 \rho_1} + \frac{1}{8 \pi 2} \int^{\alpha_2}_{\alpha_2} d \alpha^\prime_2 \left[\frac{ U^{2\beta}_{\delta_1 \rho_1} - \alpha^\prime_2U^{1\beta}_{\rho_1 \sigma_2} }{U^{0\beta}_{\rho_1 \sigma_2}}\right]\end{split}
\end{equation*}
To include math in your document, just use the math directive; here is
a simpler equation:

\begin{sphinxVerbatim}[commandchars=\\\{\}]
\PYG{o}{.}\PYG{o}{.} \PYG{n}{math}\PYG{p}{:}\PYG{p}{:}

  \PYG{n}{W}\PYG{o}{\PYGZca{}}\PYG{p}{\PYGZob{}}\PYG{l+m+mi}{3}\PYGZbs{}\PYG{n}{beta}\PYG{p}{\PYGZcb{}}\PYG{n}{\PYGZus{}}\PYG{p}{\PYGZob{}}\PYGZbs{}\PYG{n}{delta\PYGZus{}1} \PYGZbs{}\PYG{n}{rho\PYGZus{}1} \PYGZbs{}\PYG{n}{sigma\PYGZus{}2}\PYG{p}{\PYGZcb{}} \PYGZbs{}\PYG{n}{approx} \PYG{n}{U}\PYG{o}{\PYGZca{}}\PYG{p}{\PYGZob{}}\PYG{l+m+mi}{3}\PYGZbs{}\PYG{n}{beta}\PYG{p}{\PYGZcb{}}\PYG{n}{\PYGZus{}}\PYG{p}{\PYGZob{}}\PYGZbs{}\PYG{n}{delta\PYGZus{}1} \PYGZbs{}\PYG{n}{rho\PYGZus{}1}\PYG{p}{\PYGZcb{}}
\end{sphinxVerbatim}

which is rendered as
\begin{equation*}
\begin{split}W^{3\beta}_{\delta_1 \rho_1 \sigma_2} \approx U^{3\beta}_{\delta_1 \rho_1}\end{split}
\end{equation*}
Recent versions of Sphinx include built\sphinxhyphen{}in support for math.
There are three flavors:
\begin{itemize}
\item {} 
sphinx.ext.imgmath: uses dvipng to render the equation

\item {} 
sphinx.ext.mathjax: renders the math in the browser using Javascript

\item {} 
sphinx.ext.jsmath: it’s an older code, but it checks out

\end{itemize}

Additionally, matplotlib has its own math support:
\begin{itemize}
\item {} 
matplotlib.sphinxext.mathmpl

\end{itemize}

See the matplotlib \sphinxhref{https://matplotlib.org/users/mathtext.html}{mathtext guide} for lots
more information on writing mathematical expressions in matplotlib.


\section{Inserting matplotlib plots}
\label{\detokenize{chapter2:inserting-matplotlib-plots}}\label{\detokenize{chapter2:pyplots}}
Inserting automatically\sphinxhyphen{}generated plots is easy.  Simply put the
script to generate the plot in the \sphinxcode{\sphinxupquote{pyplots}} directory, and
refer to it using the \sphinxcode{\sphinxupquote{plot}} directive.  First make a
\sphinxcode{\sphinxupquote{pyplots}} directory at the top level of your project (next to
:\sphinxcode{\sphinxupquote{conf.py}}) and copy the \sphinxcode{\sphinxupquote{ellipses.py\textasciigrave{}}} file into it:

\begin{sphinxVerbatim}[commandchars=\\\{\}]
\PYG{n}{home}\PYG{p}{:}\PYG{o}{\PYGZti{}}\PYG{o}{/}\PYG{n}{tmp}\PYG{o}{/}\PYG{n}{sampledoc}\PYG{o}{\PYGZgt{}} \PYG{n}{mkdir} \PYG{n}{pyplots}
\PYG{n}{home}\PYG{p}{:}\PYG{o}{\PYGZti{}}\PYG{o}{/}\PYG{n}{tmp}\PYG{o}{/}\PYG{n}{sampledoc}\PYG{o}{\PYGZgt{}} \PYG{n}{cp} \PYG{o}{.}\PYG{o}{.}\PYG{o}{/}\PYG{n}{sampledoc\PYGZus{}tut}\PYG{o}{/}\PYG{n}{pyplots}\PYG{o}{/}\PYG{n}{ellipses}\PYG{o}{.}\PYG{n}{py} \PYG{n}{pyplots}\PYG{o}{/}
\end{sphinxVerbatim}

You can refer to this file in your sphinx documentation; by default it
will just inline the plot with links to the source and PF and high
resolution PNGS.  To also include the source code for the plot in the
document, pass the \sphinxcode{\sphinxupquote{include\sphinxhyphen{}source}} parameter:

\begin{sphinxVerbatim}[commandchars=\\\{\}]
\PYG{o}{.}\PYG{o}{.} \PYG{n}{plot}\PYG{p}{:}\PYG{p}{:} \PYG{n}{pyplots}\PYG{o}{/}\PYG{n}{ellipses}\PYG{o}{.}\PYG{n}{py}
   \PYG{p}{:}\PYG{n}{include}\PYG{o}{\PYGZhy{}}\PYG{n}{source}\PYG{p}{:}
\end{sphinxVerbatim}

In the HTML version of the document, the plot includes links to the
original source code, a high\sphinxhyphen{}resolution PNG and a PDF.  In the PDF
version of the document, the plot is included as a scalable PDF.

You can also inline code for plots directly, and the code will be
executed at documentation build time and the figure inserted into your
docs; the following code:

\begin{sphinxVerbatim}[commandchars=\\\{\}]
\PYG{o}{.}\PYG{o}{.} \PYG{n}{plot}\PYG{p}{:}\PYG{p}{:}

   \PYG{k+kn}{import} \PYG{n+nn}{matplotlib}\PYG{n+nn}{.}\PYG{n+nn}{pyplot} \PYG{k}{as} \PYG{n+nn}{plt}
   \PYG{k+kn}{import} \PYG{n+nn}{numpy} \PYG{k}{as} \PYG{n+nn}{np}
   \PYG{n}{x} \PYG{o}{=} \PYG{n}{np}\PYG{o}{.}\PYG{n}{random}\PYG{o}{.}\PYG{n}{randn}\PYG{p}{(}\PYG{l+m+mi}{1000}\PYG{p}{)}
   \PYG{n}{plt}\PYG{o}{.}\PYG{n}{hist}\PYG{p}{(} \PYG{n}{x}\PYG{p}{,} \PYG{l+m+mi}{20}\PYG{p}{)}
   \PYG{n}{plt}\PYG{o}{.}\PYG{n}{grid}\PYG{p}{(}\PYG{p}{)}
   \PYG{n}{plt}\PYG{o}{.}\PYG{n}{title}\PYG{p}{(}\PYG{l+s+sa}{r}\PYG{l+s+s1}{\PYGZsq{}}\PYG{l+s+s1}{Normal: \PYGZdl{}}\PYG{l+s+s1}{\PYGZbs{}}\PYG{l+s+s1}{mu=}\PYG{l+s+si}{\PYGZpc{}.2f}\PYG{l+s+s1}{, }\PYG{l+s+s1}{\PYGZbs{}}\PYG{l+s+s1}{sigma=}\PYG{l+s+si}{\PYGZpc{}.2f}\PYG{l+s+s1}{\PYGZdl{}}\PYG{l+s+s1}{\PYGZsq{}}\PYG{o}{\PYGZpc{}}\PYG{p}{(}\PYG{n}{x}\PYG{o}{.}\PYG{n}{mean}\PYG{p}{(}\PYG{p}{)}\PYG{p}{,} \PYG{n}{x}\PYG{o}{.}\PYG{n}{std}\PYG{p}{(}\PYG{p}{)}\PYG{p}{)}\PYG{p}{)}
   \PYG{n}{plt}\PYG{o}{.}\PYG{n}{show}\PYG{p}{(}\PYG{p}{)}
\end{sphinxVerbatim}

produces this output:

See the matplotlib \sphinxhref{https://matplotlib.org/users/pyplot\_tutorial.html}{pyplot tutorial} and
the \sphinxhref{https://matplotlib.org/gallery.html}{gallery} for
lots of examples of matplotlib plots.


\section{Inheritance diagrams}
\label{\detokenize{chapter2:inheritance-diagrams}}
Inheritance diagrams can be inserted directly into the document by
providing a list of class or module names to the
\sphinxcode{\sphinxupquote{inheritance\sphinxhyphen{}diagram}} directive.

For example:

\begin{sphinxVerbatim}[commandchars=\\\{\}]
\PYG{o}{.}\PYG{o}{.} \PYG{n}{inheritance}\PYG{o}{\PYGZhy{}}\PYG{n}{diagram}\PYG{p}{:}\PYG{p}{:} \PYG{n}{codecs}
\end{sphinxVerbatim}

produces:

See the \DUrole{xref,std,std-ref}{ipython\_directive} for a tutorial on embedding stateful,
matplotlib aware ipython sessions into your rest docs with multiline
and doctest support.


\section{This file}
\label{\detokenize{chapter2:this-file}}\label{\detokenize{chapter2:extensions-literal}}

\chapter{Indices and tables}
\label{\detokenize{index:indices-and-tables}}\begin{itemize}
\item {} 
\DUrole{xref,std,std-ref}{genindex}

\item {} 
\DUrole{xref,std,std-ref}{modindex}

\item {} 
\DUrole{xref,std,std-ref}{search}

\end{itemize}



\renewcommand{\indexname}{Index}
\printindex
\end{document}